\documentclass[a4paper]{article}
\usepackage{hyperref}
\hypersetup{
  pdftitle={HW04},
  pdfauthor={Yin Yu},
  colorlinks=true,
  linkcolor=blue,
  citecolor=cyan
}
\usepackage{amsmath}
\usepackage{listings}
\usepackage{color}
\usepackage{xcolor}
\usepackage{fullpage}
\usepackage{graphicx}
\usepackage[all,pdf]{xy}

\definecolor{mygreen}{rgb}{0,0.6,0}
\definecolor{mygray}{rgb}{0.5,0.5,0.5}
\definecolor{mymauve}{rgb}{0.58,0,0.82}

\lstset{
  backgroundcolor=\color{white},
  basicstyle=\footnotesize\ttfamily,     
  breakatwhitespace=false,      
  breaklines=true,              
  captionpos=t,   
  commentstyle=\color{mygreen}, 
  escapeinside={\%*}{*)},
  frame=t,
  keepspaces=true,
  language=Bash,
  numbers=left,                   
  numbersep=5pt,                  
  numberstyle=\tiny\color{mygray},
  rulecolor=\color{black},        
  showspaces=false,               
  showstringspaces=false,         
  showtabs=false,                 
  stringstyle=\color{mymauve},
  tabsize=4,                  
  title=\lstname,
  xleftmargin=3em,
  xrightmargin=3em
}
\usepackage{caption}
\captionsetup[lstlisting]{
  font={tt,footnotesize}
}

\title{\textbf{HW04}}
\author{\texttt{PB13011038}\quad\textbf{Yin Yu}}
\date{\today}

\begin{document}

\maketitle

\begin{enumerate}

\item[3.37] In this situation, there doesn't exist any other states
  other than these four. Four states are enough for deciding whether
  the lock should be opened.

\item[3.41] See the graph below.
  \begin{center}
    \includegraphics[width=25em]{graph/3-41.pdf}
  \end{center}

\item[3.43]
  \begin{enumerate}
    \item See the table below.
      \begin{center}
        \begin{tabular}{ccc|ccc}
          \hline
          S1 & S0 & X & D1 & D0 & Z \\
          \hline
          0 & 0 & 0 & 0 & 0 & 0 \\
          0 & 0 & 1 & 0 & 0 & 0 \\
          0 & 1 & 0 & 0 & 0 & 1 \\
          0 & 1 & 1 & 1 & 0 & 1 \\
          1 & 0 & 0 & 1 & 1 & 1 \\
          1 & 0 & 1 & 1 & 1 & 1 \\
          1 & 1 & 0 & 1 & 0 & 1 \\
          1 & 1 & 1 & 1 & 0 & 1 \\
          \hline
        \end{tabular}
      \end{center}
    \item See the diagram below.
      \begin{center}
        \includegraphics[width=25em]{graph/3-43.pdf}
      \end{center}
  \end{enumerate}

\item[3.44] It can be easily seen that:
  \begin{itemize}
  \item \verb+NOT X = X NAND X+
  \item \verb+X AND Y = NOT (X NAND Y)+
  \item \verb+X OR Y = NOT ((NOT X) NAND (NOT Y))+
  \end{itemize}
  We can build circuits according to these formulas.

\end{enumerate}

\end{document}
