\documentclass[a4paper]{article}
\usepackage{hyperref}
\hypersetup{
  pdftitle={HW01},
  pdfauthor={Yin Yu},
  colorlinks=true,
  linkcolor=blue,
  citecolor=cyan
}
\usepackage{amsmath}
\usepackage{listings}
\usepackage{color}
\usepackage{xcolor}
\usepackage{fullpage}
\usepackage[all,pdf]{xy}

\definecolor{mygreen}{rgb}{0,0.6,0}
\definecolor{mygray}{rgb}{0.5,0.5,0.5}
\definecolor{mymauve}{rgb}{0.58,0,0.82}

\lstset{
  backgroundcolor=\color{white},
  basicstyle=\footnotesize\ttfamily,     
  breakatwhitespace=false,      
  breaklines=true,              
  captionpos=t,   
  commentstyle=\color{mygreen}, 
  escapeinside={\%*}{*)},
  frame=t,
  keepspaces=true,
  language=Bash,
  numbers=left,                   
  numbersep=5pt,                  
  numberstyle=\tiny\color{mygray},
  rulecolor=\color{black},        
  showspaces=false,               
  showstringspaces=false,         
  showtabs=false,                 
  stringstyle=\color{mymauve},
  tabsize=4,                  
  title=\lstname,
  xleftmargin=3em,
  xrightmargin=3em
}
\usepackage{caption}
\captionsetup[lstlisting]{
  font={tt,footnotesize}
}

\title{\textbf{HW01}}
\author{\texttt{PB13011038}\quad\textbf{Yin Yu}}
\date{\today}

\begin{document}
\maketitle
\begin{enumerate}
\item[1.5]
\begin{enumerate}
\item
\[ \xymatrix{
  \txt{$a$}\ar[rd] & & \txt{$x$}\ar[ld] & \\
  & *+<2em>[F-,]\txt{$\times$} \ar[rd] & & \txt{$b$} \ar[ld] \\
  & & *+<2em>[F-,]\txt{$+$} \ar[d] & \\
  & & \txt{$ax+b$} &
}\]
\item
\[ \xymatrix{
  \txt{$w$}\ar[rd] & \txt{$x$}\ar[d] & & \txt{$y$}\ar[d] & \txt{$z$}\ar[ld] \\
  & *+<2em>[F-,]\txt{$+$} \ar[rd] & & *+<2em>[F-,]\txt{$+$} \ar[ld] & \\
  & & *+<2em>[F-,]\txt{$+$} \ar[rd] & \txt{$0.25$} \ar[d] & \\
  & & & *+<2em>[F-,]\txt{$\times$} \ar[d] & \\
  & & & \txt{average} &
} \]
\item
\[ \xymatrix{
  \txt{$a$}\ar[rd] & & \txt{$b$}\ar[ld] \\
  & *+<2em>[F-,]\txt{$+$} \ar[d] & \\
  & \txt{$s=a+b$} &
}
\xymatrix{
  \txt{$s$}\ar[rd] & & \txt{$s$}\ar[ld] \\
  & *+<2em>[F-,]\txt{$\times$} \ar[d] & \\
  & \txt{$a^2+2ab+b^2$} &
} \]
\end{enumerate}
\item[1.7] When things work all right, abstraction is a great
  enhancement of productivity. We can simply assume that lower-level
  structures are working correctly and use the simple abstraction
  above them to desribe or our higher-level work, without having to
  be distracted by trivial details.

  But in situations when integrating multiple components into a larger
  system, or when things go wrong, knowing the details under the
  abstraction is important and even critical. In such situations, too
  much abstraction may make it hard to make components work together,
  or hide defects or errors.
\item[1.13] Two computers should have the same ability of solving
  problems. I will prove this by showing that problems that can be
  solved by computer A can also be solved by computer B, and vice
  versa.

  Obviously, everything B can do can also be done by A, because
  B's instruction set is a subset of A's. B doesn't have substract
  instruction compared to a, but the functionality can be formed by
  adding the negative of the latter value. Therefore, everything that
  A can do can also be done by B. The two computer have the same
  computational ability.
\item[1.14] 
\begin{enumerate}
\item 120.
\item ``Bubble Sort, Pascal program, x86 ISA, Intel Core i7'';
  ``Insertion Sort, Fortran program, x86 ISA, AMD Athlon''; ``Quick
  Sort, C++ program, SPARC ISA, UltraSPARC T1''.
\item Still 120.
\end{enumerate}
\item[1.16] ISA specifies the set of instructions the computer can
  carry out, the mechanism that the computer can use to figure out
  where the oprands are located, and the number of unique locations
  that comprise the computer's memory and the number of individual 0s
  and 1s that are contained in each location.
\item[1.21] The software bought should be in \textit{assembly
    language}. It has to be in the ISA of my computer. The software
  may be written originally in high-level language but distributed in
  binary form. And since assembly languages differ on different
  computers that are in different ISAs, we must buy the software that
  uses the ISA of the computer on which the program runs.
\end{enumerate}
\end{document}
