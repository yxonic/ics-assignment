\documentclass[a4paper]{article}
\usepackage{hyperref}
\hypersetup{
  pdftitle={HW05},
  pdfauthor={Yin Yu},
  colorlinks=true,
  linkcolor=blue,
  citecolor=cyan
}
\usepackage{amsmath}
\usepackage{listings}
\usepackage{color}
\usepackage{xcolor}
\usepackage{fullpage}
\usepackage{graphicx}
\usepackage[all,pdf]{xy}

\definecolor{mygreen}{rgb}{0,0.6,0}
\definecolor{mygray}{rgb}{0.5,0.5,0.5}
\definecolor{mymauve}{rgb}{0.58,0,0.82}

\lstset{
  backgroundcolor=\color{white},
  basicstyle=\footnotesize\ttfamily,     
  breakatwhitespace=false,      
  breaklines=true,              
  captionpos=t,   
  commentstyle=\color{mygreen}, 
  escapeinside={\%*}{*)},
  frame=t,
  keepspaces=true,
  language=Bash,
  numbers=left,                   
  numbersep=5pt,                  
  numberstyle=\tiny\color{mygray},
  rulecolor=\color{black},        
  showspaces=false,               
  showstringspaces=false,         
  showtabs=false,                 
  stringstyle=\color{mymauve},
  tabsize=4,                  
  title=\lstname,
  xleftmargin=3em,
  xrightmargin=3em
}
\usepackage{caption}
\captionsetup[lstlisting]{
  font={tt,footnotesize}
}

\title{\textbf{HW05}}
\author{\texttt{PB13011038}\quad\textbf{Yin Yu}}
\date{\today}

\begin{document}

\maketitle

\begin{enumerate}

\item[4.4] Word length refers to the size of the quantities normally
  processed by the ALU. A word is handled as a unit by the instruction
  set or the hardware of the processor, therefore it's size affects
  how many bits computer can process at a time.

\item[4.5] 
\begin{enumerate}
\item Location 3: 0000000000000000

  Location 6: 1111111011010011
\item 
\begin{enumerate}
\item Location 0: 7747

  Location 1: -4059
\item Location 4: `e'
\item Location 6, 7: $8.200076\times10^{-35}$
\item Location 0: 7747

Location 1: 61477
\end{enumerate}

\item Location 0: \verb+ADD R7 R1 R3+

\item Location 5 refers to location 6. Value: 1111111011010011
\end{enumerate}

\item[4.8]
\begin{enumerate}
\item 8 bits are needed.
\item 7 bits are needed.
\item At most 3 bits.
\end{enumerate}

\item[4.10] See table below.
\begin{center}
\begin{tabular}{c|c|c|c|c|c|c}
\hline
& Fetch Instruction & Decode & Evaluate Address & Fetch Data & Execute
& Store Result \\
\hline
PC & All & - & - & - & JMP & - \\
IR & All & - & - & - & - & - \\
MAR & All & - & - & LDR & - & - \\
MDR & All & - & - & LDR & - & - \\
\hline
\end{tabular}
\end{center}

\item[4.13] IA-32 instruction takes $100\times 3 + 3 = 303$
  cycles. LC-3 takes $100 + 4 = 104$ cycles.

\item[4.15] Reinitiating RUN latch can't be done by instruction,
  because the clock stoped and no instruction can be executed. This
  can only be done by external input.

\item[4.16]
  \begin{enumerate}
  \item $5\times 10^{8}$
  \item $6.25\times 10^{7}$
  \item $5\times 10^{8}$
  \end{enumerate}

\end{enumerate}

\end{document}
