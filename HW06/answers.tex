\documentclass[a4paper]{article}
\usepackage{hyperref}
\hypersetup{
  pdftitle={HW06},
  pdfauthor={Yin Yu},
  colorlinks=true,
  linkcolor=blue,
  citecolor=cyan
}
\usepackage{amsmath}
\usepackage{listings}
\usepackage{color}
\usepackage{xcolor}
\usepackage{fullpage}
\usepackage{graphicx}
\usepackage[all,pdf]{xy}

\definecolor{mygreen}{rgb}{0,0.6,0}
\definecolor{mygray}{rgb}{0.5,0.5,0.5}
\definecolor{mymauve}{rgb}{0.58,0,0.82}

\lstset{
  backgroundcolor=\color{white},
  basicstyle=\footnotesize\ttfamily,     
  breakatwhitespace=false,      
  breaklines=true,              
  captionpos=t,   
  commentstyle=\color{mygreen}, 
  escapeinside={\%*}{*)},
  frame=t,
  keepspaces=true,
  language=Bash,
  numbers=left,                   
  numbersep=5pt,                  
  numberstyle=\tiny\color{mygray},
  rulecolor=\color{black},        
  showspaces=false,               
  showstringspaces=false,         
  showtabs=false,                 
  stringstyle=\color{mymauve},
  tabsize=4,                  
  title=\lstname,
  xleftmargin=3em,
  xrightmargin=3em
}
\usepackage{caption}
\captionsetup[lstlisting]{
  font={tt,footnotesize}
}

\title{\textbf{HW06}}
\author{\texttt{PB13011038}\quad\textbf{Yin Yu}}
\date{}

\begin{document}

\maketitle

\begin{enumerate}
\item[5.4]
  \begin{enumerate}
  \item 8
  \item 6
  \item 6
  \end{enumerate}
\item[5.9]
  \begin{enumerate}
  \item \verb+ADD R1, R1, #0+ Not NOP because it sets NZP state.
  \item \verb+BRnzp #1+ Not NOP. Jumps to the next address of
    incremented PC.
  \item Same as NOP.
  \end{enumerate}
\item[5.12] When both of the two oprands start with 1, it will
  overflow for sure. When one of the two numbers starts with 1, then
  it overflows when R2 starts with 0. Otherwise we can trust the result.
\item[5.13]
  \begin{enumerate}
  \item \verb+0001 011 010 1 00000 (ADD R3, R2, #0)+
  \item
\begin{verbatim}
1001 011 011 111111  (NOT R3, R3)
0001 011 011 1 00001 (ADD R3, R3, #1)
0001 001 010 0 00011 (ADD R1, R2, R3)
\end{verbatim}
  \item \verb+0001 001 001 1 00000 (ADD R1, R1, #0)+
  \item No. An integer is either negative or zero or positive. It
    can't have two states.
  \item \verb+0101 010 010 1 00000 (AND R2, R2, #0)+
  \end{enumerate}
\item[5.14]
\begin{verbatim}
1001 100 001 111111
1001 101 010 111111
0101 110 100 000 101
1001 011 110 111111
\end{verbatim}
\item[5.16] 
\begin{enumerate}
\item PC-relative mode is better. Because it needs only a single instruction.
\item I would prefer indirect mode. Store the memory address we want to access
  in a memory location seems clearer.
\item I would prefer base+offset mode. Offset is convient for
  continuous access.
\end{enumerate}
\item[5.27] Four. xAAAA, x30F4, x0000, x0005.
\item[5.29]
\begin{enumerate}
\item
\begin{verbatim}
LDR R2, R1, #0
STR R2, R0, #0
\end{verbatim}
\item
\begin{verbatim}
MAR <= SR
MDR <= Mem[MAR]
MAR <= DR
Mem[MAR] <= MDR
\end{verbatim}
\end{enumerate}
\item[5.30] It means $R2 = 0$; and this means $R1 = R0 - 1$. 
\end{enumerate}

\end{document}
